\begin{conclusion}

Dans cette étude, nous traitons le sujet de l'interface native du langage de programmation à objets Nit en ayant pour but de concevoir une interface native idéale pour ce langage. Tout logiciel complexe, tel que ceux comportant une interface graphique, réalisant des opérations réseau ou utilisant toute autre fonction système, se base sur une interface native directement ou indirectement. L'interface native d'un langage est donc largement utilisée, sa qualité influence directement celle de tout logiciel réalisé en ce langage.

\section*{Travaux futurs}

Le concept de l'interface native du langage Nit peut être adapté pour tout autre langage de programmation à objets. Une interface native semblable pourrait être réalisée pour d'autres langages permettant ainsi de bénéficier d'une syntaxe native naturelle et de la sûreté d'exécution amenée par nos contributions.

\end{conclusion}
