% contenu:
% sujet et sa portée
% état de la question et problème a resoudre
% objectifs
% méthode utilisée
% démarche adoptée
% structure du document

\begin{introduction}
\label{introduction}

Les langages de programmation de haut niveau, tel que Java~\cite{java}, Python~\cite{pythonRef}, Ruby~\cite{rubyCookbook}, C\#~\cite{csharp} et beaucoup d'autres, sont loin du fonctionnement de la machine physique, simple d'utilisation et offrent aux programmeurs plusieurs services tels qu'un paradigme de programmation avancé, une syntaxe expressive, une gestion automatique de la mémoire ou encore, une assurance de sûreté d'exécution. Le langage de programmation à objets Nit~\cite{nit} est de cette catégorie, il utilise une syntaxe simple, comporte un ramasse-miettes et offre des optimisations statiques à la compilation. En comparaison, le langage C~\cite{ansi-c89} conserve de forts avantages, il permet la programmation près de la machine et l'accès à une grande quantité de code préexistant. 

\section{Langage de programmation Nit}

Nous avons réalisé cette étude dans le cadre du développement du langage de programmation Nit~\cite{nit}. Dans cette section, nous présentons les particularités du langage Nit principalement en rapport avec les autres langages de programmation à objets. Ces particularités sont utiles pour comprendre les décisions que nous avons prises au cours de l'étude et nous leur faisons référence à répétition dans ce document.

\end{introduction}
